\documentclass[norsk, 12pt]{article}
\usepackage[norsk]{babel} 
\usepackage[utf8]{inputenc} 
\usepackage{times}			% Default times font style
\usepackage[T1]{fontenc} 	% Font encoding
\usepackage{amsmath} 		% Math package
\usepackage{amssymb} 		% Math symbols package

\usepackage{hyperref}
\usepackage[letterpaper, margin=1in]{geometry}

\newcommand{\la}{\langle}
\newcommand{\ra}{\rangle}
\newcommand{\vm}[1]{\begin{vmatrix}#1\end{vmatrix}}

\title{}
\begin{document}
\section*{Exercise 1.3}

Given wave functions

\begin{align*}
	\Psi(\vec R) \in X= \mathbb{R}^3
\end{align*}

\subsection*{1.3.1}
We first consider
\begin{align*}
	\Psi(\vec r_1, \vec r_2) = e^{-\alpha |\vec r_1 - \vec r_2|} = e^{-\alpha r_{21}}
\end{align*}

Permuting the particles in this wave function, we see that we get
\begin{align*}
	P_{\sigma}\Psi = e^{-\alpha |\vec r_2 - \vec r_1|} = e^{-\alpha r_{12}} = \Psi
\end{align*}

Since $r_{12}=r_{21}$ is just the distance between the particles,
we see that the eigenvalue of $P_{\sigma}$ must be 1, and thus $\Psi$
is \textit{totally symmetric}.

We also have that $\Psi$ is square integrable for $\alpha \neq 0$, since
\begin{align*}
	\int_{-\infty}^{\infty} \mathrm{d}x\: |f(x)|^2 =
	\int_{-\infty}^{\infty} \mathrm{d}x\: e^{-c^*|x| - c|x|}=
	\int_{-\infty}^{\infty} \mathrm{d}x\: e^{-2a|x|}=\frac{1}{a}<\infty
\end{align*}
for $c = a + ib$.
\subsection*{1.3.2}
If we now permute
\begin{align*}
	\Psi(\vec r_1, \vec r_2) = \sin(\vec e_z\cdot(\vec r_1 - \vec r_2)) = \sin(z_1-z_2)
\end{align*}
instead, we get
\begin{align*}
	P_{\sigma}\Psi = \sin(z_2-z_1) = -\sin(z_1-z_2) = (-1)^1 \Psi
\end{align*}
which means that the wave function is \textit{totally antisymmetric}.

This function is not square integrable, since
\begin{align*}
	\int_{-\infty}^{\infty} \mathrm{d}x\: \sin^2(x) = \infty
\end{align*}

\subsection*{1.3.3}
Finally, we look at
\begin{align*}
	\Psi(\vec r_1, \vec r_2, \vec r_3) =
	\sin[\vec r_1 \cdot(\vec r_2\times \vec r_3)]
	\prod_{i=1}^{3}e^{-|\vec r_i|^2}
\end{align*}
where we see that the product part can be permuted without consequences to the
eigenvalue.

For the \textit{scalar triple product}, we have
\begin{align*}
	&\vec i\cdot(\vec j\times\vec k) = \vec k\cdot(\vec i\times\vec j)
	=\vec j\cdot(\vec k\times\vec i)\\
	=-&\vec j\cdot(\vec i\times\vec k)=-\vec k\cdot(\vec j\times\vec i)=
	-\vec i\cdot(\vec k\times\vec j)
\end{align*}
The upper part here corresponds to \textit{even} number of permutations,
and the lower part corresponds to \textit{odd} number of permutations.

Thus
\begin{align*}
    P_{\sigma}\Psi = (-1)^{|\sigma|}\Psi
\end{align*}
and the function is \textit{totally antisymmetric}.

The 



\section*{Exercise 1.13}
\label{sec:2}
\subsection*{a)}
\label{sub:1.13a}

Given the subspace $L^2(X^N)_{AS}$,
we want to find a basis for $N=2,3,4$ particles, when we have the orthonormal orbitals
$\phi_{\mu},\:\mu=1,…,6$.

Such a basis can be written as a Slater determinant
\begin{align*}
    \Phi_{\mu_1,...,\mu_N} &= \la \vec x| \vec\mu\ra 
    = \la x_1\cdots x_N | \mu_1\cdots\mu_N \ra\\
    &=\frac{1}{\sqrt{N!}}
    \vm{\phi_{\mu_1}^1&\cdots&\phi_{\mu_1}^N\\
    \vdots&\ddots&\vdots\\
    \phi_{\mu_N}^1&\cdots&\phi_{\mu_N}^N}
\end{align*}

\paragraph{N=2}
\label{par:1}
The basis for two particles is then
\begin{align*}
	\Phi_{\mu_1,\mu_2} &= \la x_1 x_2 | \mu_1 \mu _2 \ra\\
    &=\frac{1}{\sqrt{2}}\left(\phi_{\mu_1}^1 \phi_{\mu_2}^2 - \phi_{\mu_2}^1 \phi_{\mu_1}^2 \right)
\end{align*}

\paragraph{N=3}
\label{par:2}
\begin{align*}
	\Phi_{\mu_1,\mu_2,\mu_3} &= \la x_1 x_2 x_3 | \mu_1 \mu _2 \mu_3 \ra\\
    &=\phi_{\mu_1}\la x_2 x_3 | \mu _2 \mu_3 \ra - \phi_{\mu_2}\la x_1 x_3 | \mu _1 \mu_3 \ra + \phi_{\mu_3}\la x_1 x_2 | \mu _1 \mu_2 \ra
\end{align*}

\paragraph{N=4}
\label{par:3}
\begin{align*}
	\Phi_{\mu_1,\mu_2,\mu_3,\mu_4} &= \la x_1 x_2 x_3 x_4 | \mu_1 \mu _2 \mu_3 \mu_4 \ra\\
    &=\phi_{\mu_1}\la x_2 x_3 x_4 | \mu _2 \mu_3 \mu_4 \ra 
    - \phi_{\mu_2}\la x_1 x_3 x_4 | \mu _1 \mu_3 \mu_4 \ra\\&
    + \phi_{\mu_3}\la x_1 x_2 x_4 | \mu _1 \mu_2 \mu_4 \ra
    -  \phi_{\mu_4}\la x_1 x_2 x_3 | \mu _1 \mu_2 \mu_3 \ra\\
\end{align*}

\subsection*{b)}
\label{sub:1.13b}

We can only have six particles when we have six one-particle functions. 
If we filled up the determinant with more of the same one-particle functions, we
could get equal rows, and the determinant would be zero.

\subsection*{c)}
\label{sub:1.13c}
We have
\begin{align*}
	|\vec\mu\ra = |\mu_1\cdots \mu_N\ra
\end{align*}
and $\mu = 1,...,6$, which means we have six bits that can be on or off
\begin{align*}
	|n_1\cdots n_6\ra
\end{align*}
\paragraph{N=2}
For only two particles, we get $\{\mu_1,\mu_2\} = \{1,2\}$, and
\begin{align*}
	|\mu_1 \mu_2\ra = |1,2\ra = |110\ra
\end{align*}
since
\begin{align*}
	110_2 = 2^2 + 2^1
\end{align*}

\paragraph{N=3}
\begin{align*}
	|\mu_1 \mu_2 \mu_3\ra = |1,2,3\ra &= |1110\ra
\end{align*}

\paragraph{N=4}
\begin{align*}
	|\mu_1 \mu_2 \mu_3 \mu_4\ra = |1,2,3,4\ra &= |11110\ra
\end{align*}

\subsection*{d)}
\label{sub:1.13d}
Since the Fock space is the direct sum of all $L^2_N$ spaces
\begin{align*}
	\dim F = \sum_{N=0}^{\infty}\dim L^2(X^N) = \sum_{i=1}^{N} 6^i
\end{align*}

\subsection*{e)}
\label{sub:1.13e}
Since we have $L$ orbitals, we get
\begin{align*}
	\dim F = \sum_{i=1}^{N} L^i
\end{align*}

\end{document}