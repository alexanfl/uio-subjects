\documentclass[norsk, 12pt]{article}
\usepackage[norsk]{babel} 
\usepackage[utf8]{inputenc} 
\usepackage{times}			% Default times font style
\usepackage[T1]{fontenc} 	% Font encoding
\usepackage{amsmath} 		% Math package

\usepackage{hyperref}
\usepackage[letterpaper, margin=1in]{geometry}
\usepackage{parskip}		% Norske avsnitt

\title{}
\begin{document}
\begin{align*}
	H(0) \cdot \Delta e
\end{align*}

Pulseksitasjon
\begin{align*}
	r(t) &= \frac{H(0)\Delta e}{T_1-T_2}\int_{0}^{\epsilon}\left \{ e^{/(t-\tau)/T_1} - e^{/(t-\tau)/T_1} d\tau \right\}\\
	\Rightarrow r_p(t) &= H(0)\cdot \Delta e (e^{\epsilon/T_{1,2}}-1)e^{-t_p/T_{1,2}}
\end{align*}

\section{Example: Harmonic dampened oscillator}
\begin{align*}
	m\ddot{r} &= -\mu \dot{r}(t) - k r(t) + e(t)\\
	\left( ms^2 + s\mu + k \right) R(s) &= E(s) \:\leftarrow \text{Laplace transform}\\
	R(s) = \frac{1}{ms^2 + s\mu + k} E(s) &= \frac{E(s)}{H(s)}\\
	s_{1,2} &= \frac{-\mu \pm \sqrt{\mu^2 - 4km}}{2m}
\end{align*}

Realdelen til polene er negative.

Consider the step excitation 
\begin{align*}
	R(s) = \frac{\Delta e}{m(s-s_1)(s-s_2)s}
\end{align*}
Use partial fractions on that gives
\begin{align*}
	R(s) = \frac{\Delta e}{m(s_1-s_2)(s-s_1)s_1} + \frac{\Delta e}{m(s_2-s_1)(s-s_2)s_2}
	+ \frac{\Delta e}{ms_1 s_2 s}
\end{align*}
\begin{align*}
	\int_{0}^{\infty}e^{s_1 t} e^{-s t} dt = \frac{1}{s-s_1}
\end{align*}
This gives that
\begin{align*}
	r(t) = \frac{\Delta e}{m(s_1-s_2)s_1}e^{s_1 t} + \frac{\Delta e}{m(s_2-s_1)s_2} e^{s_2 t}
	+ \frac{\Delta e}{ms_1 s_2}
\end{align*}

\subsection{$t=0$}
\begin{align*}
	r(0) = \frac{\Delta s_2 - \Delta e s_1 + \Delta e (s_1-s_2)}{m(s_1-s_2)s_1 s_2} = 0
\end{align*}

\subsection{$t=\infty$}
\begin{align*}
	r(\infty) = \frac{\Delta e}{k}
\end{align*}
Assume that
\begin{align*}
	\frac{k}{m} > \frac{\mu^2}{4m^2}
\end{align*}
Then
\begin{align*}
	\frac{1}{m(s_1-s_2)s_1} &= |M|e^{i\phi}\\
	\frac{1}{m(s_2-s_1)s_2} &= |M|e^{-i\phi}
\end{align*}
Gives that
\begin{align*}
	r(t) = 2\Delta e |M| \cos(\omega t + \phi)e^{-\alpha t} + \frac{\Delta e}{k}
\end{align*}

\begin{align*}
	s_1 = -\alpha + i\omega
\end{align*}

\end{document}