\documentclass[norsk, 12pt]{article}
\usepackage[norsk]{babel} 
\usepackage[utf8]{inputenc} 
\usepackage{times}			% Default times font style
\usepackage[T1]{fontenc} 	% Font encoding
\usepackage{amsmath} 		% Math package

\usepackage{hyperref}
\usepackage[letterpaper, margin=1in]{geometry}
\usepackage{parskip}		% Norske avsnitt

\title{}

\begin{document}

Første forelesning i FYS4180 – Eksperimentelle metoder i fysikk

  - Målefeil: konstante og systematise avvik, gjentar seg på samme måte. Kan ofte korrigeres.

  - Måleusikkerhet: avvik av statistisk natur. Fordi resultatene er usystematiske, kan vi ikke forutsi avviket.
    Kan gjøre flere eksperimenter for å finne en bedre middelverdi.
    Kan ofte reduseres ved å definere måleobjektet skarpere.
    Angi usikkerheten i siste desimal.

  Fordelingsfunksjoner og histogrammer
    - Deler opp i intervaller (bins) av dx
    - Hvor mange målinger finner man i det intervallet.
    - f_k er antallet målinger i [x_k, x_k + dx]
    - Etter N målinger, N = \sum_{k} f_k
    - Får ut et histogram.
    - Når man har uendelig god oppløsning, får man fordelingsfunksjonen:
      
      p(x) = \lim_{N\rightarrow\infty} \lim_{dx\rightarrow 0} f_k/(N dx)
    
    - p(x)dx er sannsynligheten for å finne en måling i intervallet [x, x + dx]
    - p(x) er normert.
    - Definerer en middelveri, gitt uendelig god oppløsning:
    
      <f(x)> = \int_{-infty}^{infty} f(x) p(x) dx

    - «sann verdi» av måling: X \equiv <x>, der <x> = \int_{-infty}^{infty} x p(x) dx.
    
    - Varians til fordelingsfunksjon:

      \sigma^2 = \int_{-infty}^{infty} (x - X) p(x) dx

    - Standardavviket er \sigma.

    - Gjentagelsesmålinger
      - x_1, …, x_n
      - Definerer middelverdien av målingene:
        
        \xbar = \frac{1}{n}\sum_{i=1}^n x_i

      - Avvikk av enkeltmåling fra sann verdi:
        
        e_i = x_i - X

      - Avviket av middelverdi fra sann verdi:

        E = \xbar - X

        E = \frac{1}{n}\sum_i (x_i - X) = \frac{1}{n}\sum e_i

        E^2 = \frac{1}{n^2}\sum_i e_i + \frac{1}{n^2}\sum_i \sum_{j\neq i} e_i e_j

        <E^2> = \frac{1}{n^2}\sum <e_i^2> + \frac{1}{n^2} \sum_{j\neq i} <e_i e_j>

Hvis e_i og e_j er uavhengige:
      
\begin{equation}
    <e_i e_j> = <e_i><e_j>\\
    <e_i> = <x_i - X> = <x_i> - X = X - X = 0
\end{equation}
    
begin{equation}
	\sigma_m^2 \equiv <E^2> = <(\xbar- X)^2>\\
	\sigma^2 = <e^2>
\end{equation}

PUTT INN RESTEN AV VERDIENE FRA SQUIERES HER

\begin{equation}
	s^2 \frac{1}{n}\sum_i (x_i - \xbar)^2
	= \frac{1}{n}\sum_i (e_i - E)^2
	= \frac{1}{n}\sum_i e_i^2 - \frac{2}{n}\sum_i e_i E - \frac{1}{n} - sum_i E^2
\end{equation}

\end{document}