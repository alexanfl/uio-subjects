\documentclass[norsk, 12pt]{article}
\usepackage[norsk]{babel} 
\usepackage[utf8]{inputenc} 
\usepackage{times}			% Default times font style
\usepackage[T1]{fontenc} 	% Font encoding
\usepackage{amsmath} 		% Math package
\usepackage{parskip}

\usepackage{hyperref}
\usepackage[letterpaper, margin=1in]{geometry}
%\usepackage{parskip}		% Norske avsnitt

\title{}
\begin{document}
\section{Granulære materialer}
    Gruppa til Knut gjør både simuleringer og eksperimenter.
	
    \textit{Eksperimentelt:} glycerin og partikler på glassplate. Danner labyrintmønster.
    Vil ligge et lag med partikler på bunnen av glassplaten. Lufttrykk presser 
    på partiklene.
    Får en raseffekt/-front mot væsken.
    Pumper langsomt for å unngå viskøst trykkfall.
    Nesten alt trykkfall på fronten.

\end{document}