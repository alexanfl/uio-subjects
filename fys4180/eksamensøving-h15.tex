\documentclass[norsk, 12pt]{article}
\usepackage[norsk]{babel} 
\usepackage[utf8]{inputenc} 
\usepackage{times}			% Default times font style
\usepackage[T1]{fontenc} 	% Font encoding
\usepackage{amsmath} 		% Math package

\usepackage{hyperref}
\usepackage[letterpaper, margin=1in]{geometry}
\usepackage{parskip}		% Norske avsnitt

\title{}
\begin{document}
\section{Transmisjonselektronmikroskop}
\label{sec:1}
\subsection{Hvordan TEM virker}
\label{sub:1.1}
Elektroner blir skutt som en stråle mot en 
prøve og på veien blir den konsentrert
av en elektromagnetisk linse (konveks).

Når elektronene treffer prøven,
kan det skje forskjellige ting
med elektronene,
men det TEM ser på er de elektronene
som transmitteres eller som
blir elastisk spredt.

Elektronlinsen består av en
jernring med eksitasjonsspoler
som gjør at det genereres et
magnetfelt med rotasjonssymmetri.
Feltlinjene er fordelt konvekst
med hensyn til den optiske
aksen til strålen.

Det er tre typer bildekontraster
som kan oppstå i TEM.

\paragraph{Spredningskontrast}
\label{par:1.1.1}
Skjer når man setter inn
en liten «bore» som gjør at kun
de spredte elektronene med liten vinkel
kommer gjennom. Resten blokkeres.
\paragraph{Diffraksjonskontrast}
\label{par:1.1.2}
Når elektronene spres elastisk,
kan de forsterke hverandre (Bragg-
refleksjon)
\paragraph{Fasekontrast}
\label{par:1.1.3}
Oppstår på grunn av små faseforskjeller
mellom spredete og transmitterte
eletroner.


\end{document}