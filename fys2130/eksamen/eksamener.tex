\documentclass[12pt]{article}
\usepackage[norsk]{babel} 
\usepackage[utf8]{inputenc} 
\usepackage[T1]{fontenc} 	% Font encoding
\usepackage{amsmath} 		% Math package
\usepackage{mathtools} 		% Adds the declare paired 
							% delimeter command to make costom \abs and \norm
\usepackage{breqn}		 	% Adds dmath environment for automated brakeline
\usepackage{xfrac}			% Adds slanted fractions (sfrac)
\usepackage{cancel}			% Adds the cancel command, a slash through the symbol(s)
\usepackage{tabularx}		% Adds adjustable width on tabulars
\usepackage{cuted}			% Adds the strip command, pagewidth text in a twocolumn
							% environment.  
\usepackage{hyperref}
\usepackage{parskip}		% Norske avsnitt
\usepackage{siunitx}		% SI-enheter
\usepackage{tablefootnote}
\usepackage{tabularx}
\usepackage{listings}

\usepackage{graphicx}
\usepackage{caption}
\usepackage{subcaption}

% Alghorithm packages:
\usepackage{algorithm}
\usepackage[noend]{algpseudocode}

% colored box
\usepackage[most]{tcolorbox}

% Numbers within section
\numberwithin{equation}{section}
\numberwithin{figure}{section}

% Start custom \abs \norm 
\DeclarePairedDelimiter\abs{\lvert}{\rvert}%
\DeclarePairedDelimiter\norm{\lVert}{\rVert}%

\newcommand{\equ}[1]{{\small\begin{align*}#1\end{align*}}}
\newcommand{\ita}[1]{\textit{#1}}
\newcommand{\ma}[1]{$#1$}
\newcommand{\eq}[1]{{\small\begin{align}#1\end{align}}}
\newcommand{\mat}[1]{\begin{matrix}#1\end{matrix}}
\renewcommand\vec[1]{\mathbf{#1}}
\newcommand{\OP}[1]{\mathbf{\widehat{#1}}}
\newcommand{\op}[1]{\hat{#1}}
\newcommand{\unit}[1]{\mathbf{\hat{#1}}}

\begin{document}
\pagenumbering{gobble}

\tcbset{width=\textwidth,colback=lime!25!white,colframe=green!40!black,
arc=0pt,outer arc=0pt}
\section{Eksamen 2014}
\begin{tcolorbox}[title = Kinematikk]
  En fjærbevegelse beskrives med ligningen 
  \eq{\ddot z + (b/m)\dot z + (k/m)z = 0} 
  der $z$ avhenger av tiden.
  
  Den utledes ved å se på Hooks lov $F_H = -kz$, og et friksjonsledd
  $F_f = -b\dot z$, som er avhengig av hastigheten og en konstant.
  Når man ser dette i sammenheng med N2L, får vi ligning 1.1.
  \tcbline
  I en svingeligning må vi ha med det første og siste leddet i 1.1.
  Fordi det midterste leddet bare bestemmer dempingen.
  \tcbline
  Det finnes tre kategorier av løsninger for 1.1
  \eq{z = A e^{\alpha t}}
  der
  \eq{\alpha = -\gamma \pm \sqrt{\gamma^2 - \omega^2}}
  
  Overkritisk demping: $\gamma > \omega$
  
  Kritisk demping: $\gamma = \omega$
  
  Underkritisk demping: $\gamma < \omega$
  
  For kritisk demping blir løsningen
  
  \eq{z = Ae^{-\gamma t} + Bte^{-\gamma t}}
  \tcbline
  For en ytre, tidsvariabel kraft som påvirker systemet vårt,
  har vi en \ita{faseresonansfrekvens}
  \eq{f_0 = \frac{\omega}{2\pi}}
  
  Dette er frekvensen systemet ville svingt med uten friksjvinon ($b=0$).
  \tcbline
  Siden systemet vårt etterhvert vil svinge med frekvensen til den ytre kraften,
  ville vi ikke kunne se spor av de tre kateogoriene av løsninger ovenfor.
  
  Derimot vil vi kunne se slike karakteristikker hvis vi varierer frekvensen
  og beregner \ma Q-faktoren, eller ser på systemet like etter at den påtrykte kraften 
  er satt i gang.
\end{tcolorbox}

\tcbset{width=\textwidth,colback=cyan!25!white,colframe=cyan!75!black,
arc=0pt,outer arc=0pt}
\begin{tcolorbox}[title = Lys gjennom glassprisme]
  Hvis vi sender lys fra luft inn i et glassprisme (se figur \ref{fig:1a}), 
  der lyset har polarisasjon vinkelrett på
  overflaten, vil deler av lyset bli transmitert 
  og resten reflektert (se figur \ref{fig:1b}).
  Innfallsvinkelen $\theta_i$ er vinkelen mellom innfallslodd (vinkelrett på overflaten)
  og strålen. Denne er lik refleksjonsvinkelen $\theta_r$.
  Fra Snels lov har vi da
  \eq{n_t \theta_t = n_i \theta_i}
  der $n_i$ og $n_t$ er brytningsindeksen til hhv. luft og glass.
  \tcbline
  Når lyset går gjennom glasset vil glassatomene bli påvirket av et
  elektrisk felt, slik at elektronene oscillerer om atomkjernen.
  det gir en netto polarisering som genererer en ny bølge som brer seg i samme retning
  (se figur \ref{fig:1c}).
  
  Summen av de to bølgene gjør at lysbølgen går saktere enn gjennom vakuum.
  Den relative elektriske permitiviteten $\epsilon_r$ forteller hvor
  lett et materiale polariseres. Lyshastigheten gis dermed som
  \eq{c_{glass} = \frac{c_0}{\epsilon_{r,\;glass}}}
  
  Merk at når vi snakker om polarisering av mediet, er det i en annen betydning av ordet
polarisering enn når vi angir retningen på det elektriske feltet i en elektromagnetisk bølge.
\tcbline

\end{tcolorbox}

\begin{figure}[h!]
  \caption{}
  \begin{subfigure}[b]{0.5\textwidth}
  	\includegraphics[width=60mm]{bilder-eksamener/lys1.png}
  	\caption{Lys mot glassprisme}\label{fig:1a}
  \end{subfigure}
    \begin{subfigure}[b]{0.5\textwidth}
  	\includegraphics[width=60mm]{bilder-eksamener/lys2.png}
  	  	\caption{Refleksjon og transmisjon}\label{fig:1b}
  \end{subfigure}\\
    \begin{subfigure}[b]{\textwidth}
  	\includegraphics[width=\textwidth]{bilder-eksamener/lys3.png}
  	  	\caption{Polarisering av atomer når en elektromagnetisk bølge passerer (pga det elektriske
feltet i bølgen).}\label{fig:1c}
  \end{subfigure}  
\end{figure}



\end{document}