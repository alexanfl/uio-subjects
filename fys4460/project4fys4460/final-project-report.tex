\documentclass[12pt]{article}
\usepackage[utf8]{inputenc} 
\usepackage[T1]{fontenc} 	% Font encoding
\usepackage{amsmath} 		% Math package
\usepackage{mathtools} 		% Adds the declare paired 
							% delimeter command to make costom \abs and \norm
\usepackage{breqn}		 	% Adds dmath environment for automated brakeline
\usepackage{xfrac}			% Adds slanted fractions (sfrac)
\usepackage{cancel}			% Adds the cancel command, a slash through the symbol(s)
\usepackage{tabularx}		% Adds adjustable width on tabulars
\usepackage{tabu}           % colored font in tabular
							% environment.  
\usepackage{hyperref}
\usepackage{siunitx}		% SI-enheter
\usepackage{times}
\usepackage[dvipsnames]{xcolor,colortbl} % expanded colors and table colors

% Capitalized headings
\usepackage{sectsty}
\usepackage{parskip}

% Set margins
\usepackage[a4paper, margin=0.75in]{geometry}


% Numbers within section
\numberwithin{equation}{section}
\numberwithin{figure}{section}

% Start custom \abs \norm 
\DeclarePairedDelimiter\abs{\lvert}{\rvert}%
\DeclarePairedDelimiter\norm{\lVert}{\rVert}%

\newcommand{\equ}[1]{{\small\begin{align*}#1\end{align*}}}
\newcommand{\mc}[1]{\mathcal{#1}}
\newcommand{\ita}[1]{\textit{#1}}
\newcommand{\ma}[1]{$#1$}
\newcommand{\eq}[1]{{\small\begin{align}#1\end{align}}}
\newcommand{\mat}[1]{\begin{matrix}#1\end{matrix}}
\renewcommand\vec[1]{\mathbf{#1}}
\renewcommand\vec[1]{\mathbf{#1}}
\renewcommand\vec[1]{\mathbf{#1}}
\newcommand{\OP}[1]{\mathbf{\widehat{#1}}}
\newcommand{\op}[1]{\hat{#1}}
\newcommand{\unit}[1]{\mathbf{\hat{#1}}}



%% Section renewcomands
\renewcommand{\thesection}{\arabic{section}.}
\renewcommand{\thesubsection}{\arabic{subsection}.}
\renewcommand{\thesubsubsection}{\arabic{subsection}.\arabic{subsubsection}}
\renewcommand{\theequation}{\arabic{section}.\arabic{equation}}
\renewcommand{\thefigure}{\arabic{subsection}.\arabic{figure}}




\title{\Huge\bfseries \MakeUppercase{Final Project---FYS4460} \\[12pt]
    \large Introducing the Nosé-Hoover Thermostat in Lennard-Jones Systems}
\author{Alexander Fleischer}


\begin{document}

\maketitle
\begin{center}
\ita{Introduction of the Nosé-Hoover thermostat
    and measurements of RMS-displacement using various
    thermostats. Discussion of the various thermostats for
    for various thermodynamic measures.}
    
\end{center}



\sectionfont{\bfseries\normalsize\MakeUppercase}


\section{Introduction}
In this final project, I've extended the studies of the Lennard-Jones
system of Argon atoms, that we started in the first project. 
In particular, I've looked at three different thermostats---the
Anderson, Berendsen and Nosé-Hoover thermostats.

Statistical mechanics is all about predicting the
thermodynamic properties of a system.
These properties could be the equations of 
state\footnote{A thermodynamic equation that relates
two or more of the system states, such as temperature, internal
energy, pressure and volume.},
phase transition\footnote{Going from for example solid to fluid.} 
of the matter and so on.
Studying such systems analytically often proves difficult,
unless we are looking at idealizes, simple 
systems\footnote{Such systems could be the Ising model,
    ideal paramagnets, Einstein solids etc.}.
Luckily, we can study the realistic systems through numerical
models. Molecular dynamics (MD) is a powerful tool
to study the real systems more closely.

In a classical MD simulation we are often interested
in studying the \ita{microcanonical ensemble}.
The ensemble is all the configurations a system can have
under the conditions of constant number of particles $N$,
volume $V$ and total energy $E$.
We therefore often refer to it as the $NVE$ ensemble.
A box of volume $V$ with $N$ particles that are
unaffected by the surroundings is such an ensemble,
and it can obviously have an enormous amount of different sets of
states.
Over time, we expect that all the particles microstates are
equally likely to occur. This is called the
\ita{fundamental assumption of statistical mechanics}.
If we now take the average of the whole ensemble and all its
configurations, we can calculate the thermodynamic properties
of the system.

For large and complicated systems, we will have difficulties
calculating the ensemble average.
Instead we simulate the system and its evolution over time
with molecular dynamics methods.
The reason for this is that the time-evolution average of our
system should give us an estimate of the ensemble average.

\section{From the Microcanonical to the Canonical Ensemble}
The microcanonical ensemble had a constant energy since we didn't
let any energy leave or enter the system.
While the energy is constant, there would still be fluctuations
in the temperature. By stabilizing the temperature, we go
from microcanonical to the \ita{canonical} ensemble.
To make the temperature constant, we couple it to a 
heat bath of some desired temperature.
This is called a thermostat.


Obviously energy will travel from or to the heat bath and
the total energy is no longer constant---now we have
a $NVT$ ensemble.

For a finite system, though we claim to have a $NVT$ ensemble,
we will see some temperature fluctuations.
We can find averages and variances from the Boltzmann distribution,
and the system fluctuates as
\begin{equation}
    \frac{\sigma_T}{\langle T \rangle} = \sqrt{\frac{2}{3N}}
\end{equation}

This tells us that the system size determines the size of the fluctuations.
As $N\rightarrow \infty$, the fluctuations fade away.
A consequence is that the energy dies out too,
and the microcanonical and canonical ensemble merge together.

To couple the system to the heat bath and creating the canonical ensemble,
we will look at three thermostats: the Berendsen, Andersen and Nosé-Hoover
thermostats.

A thermostat should satisfy some requirements:
\begin{itemize}
    \item It should be able to control the temperature of the system
        such that it is close to the temperature of the heat bath.
    \item It should not disturb the system dynamics, and it should
        sample the quantities of the phase space (positions and velocities).
    \item It should be adjustable.
    \end{itemize}

\section{Temperature Measurements}
To measure the temperature of the system in the 
$NVE$ ensemble, we use the equipartition theorem.
The equipartion theorem states that at thermal equilibrium,
every quadratic degree of freedom in the Hamiltonian of the system
must have an average energy of $\sfrac{1}{2}k_b T$.
Every atom in the system has three quadratic degrees of freedom in the
kinetic energy, and there are $N$ atoms, thus at equilibrium the kinetic 
energy is
\begin{equation}
    \langle E_k \rangle = \frac{3}{2}N k_b T
\end{equation} 

This is the ensemble average of the kinetic energy,
and as we saw earlier, the time-average for our system should be the same.
The estimated temperature is then
\begin{equation}
    T = \frac{2}{3} \frac{\langle E_k \rangle}{N k_b}
\end{equation}


\section{The Berendsen Thermostat}
Many thermostats rescale the velocities of the systems' particles
by a factor $\gamma$. One of these is the \ita{Berendsen thermostat}.
The scaling factor is
\begin{equation}
    \gamma = \sqrt{1+\frac{\Delta t}{\tau}\left(
            \frac{T_{\text{bath}}}{T} - 1    
        \right)}
\end{equation}
Here $\tau$ is called the \ita{relaxation time}: which lets us
adjust the (weak) coupling to the heat bath. A greater $\tau$ value gives
a weaker coupling.
This means that the Berendsen thermostat works by dampening the
fluctuations of the systems' kinetic energy.
The velocity scaling is such that change in temperature is proportional
to difference in temperature between the heat bath and the system.
\begin{equation}
    \Delta T = \frac{\Delta t}{\tau}(T_0 - T(t))
\end{equation}

The thermostat is not able to represent a correct canonical ensemble,
but for large enough systems, we will achieve approximately
correct results.
It is often used because it effectively thermalizes the system to
the temperature of the heat bath,
and can be used to initalize a system before applying the Nosé-Hoover
thermostat.

\section{The Andersen Thermostat}
Another simple thermostat is the \ita{Andersen thermostat}.
It works by simulating collisions between particles in the system
and in the heat bath.
For every particle in the system we generate a random number
uniformly generated between 0 and 1.
If the number is less than $\sfrac{\Delta t}{\tau}$,
we let the particle be one of the colliding particle. 
Thus the particle needs a new velocity, which is
generated from a Boltzmann distribution with standard deviation
$\sqrt{k_B T_{\text{bath}}/m}$.
We can adjust the thermostat by changing $\tau$, which can
be considered the \ita{collision time}.
The collision time describes the strength of the coupling beween
the system and the heat bath.

Assuming ergodicity, the Andersen thermostat gives us
a canonical distribution of microstates.
The Algorithm is non-deterministic and thus not time-reversible.
It should only be used when measuring time-independent properties.
Diffusion, which is something we have looked at, should not be
calculated if the system has been influenced by the Andersen thermostat.

The thermostat is useful for equilibrating the system,
but disturbs the natural vibrations of the crystal structure.

In project 1 we experienced that the Berendsen thermostat gave smaller
fluctuations in the system than the Andersen thermostat.



\section{The Nosé-Hover Thermostat}
Our goal in this project was to implement a more complicated and
better thermostat. This is the Nosé-Hoover thermostat,
and we wish to compare it to the two others.
The Nosé-Hoover thermostat introduces the heat bath as an additional pair
of degrees of freedom to the system---making the heat bath part of our
system. It is done by introducing the degrees of freedom in the
systems Lagrangian. It is done in such a way that when the total
acts like a micro-canonical ensemble, the actual degrees
of freedom samples a canonical ensemble. 
It relies on that we introduce the virtual mass $Q>0$
and the virtual velocity $\dot{\tilde s}$. The Lagrangian, 

\begin{equation}
    \mc L = \sum_i \frac{1}{2}m_i{(s\dot{r}_i)}^2 - u(r) 
    + \frac{1}{2}Q\dot{s}^2 - gk_b T_0 \ln s
\end{equation}

The first two terms are the kinetic and potential energy of
the actual system, while the two last are the kinetic
and potential energy of $\tilde s$.
Here $g$ is $3N + 1$ in the Nosé formalism. This uses
$\tilde s$ as a time-scaling parameter, and the actual time step is
not constant. In the Hoover formalism, $g=3N$, and there is no time
scaling. This means the time-step is constant.
Then we can choose a friction parameter $\gamma = \dot s/s$.
Now we don't have to solve the equations of motion for
for the velocity (or momentum), and only include
the friction in the equations. In MD units we then have
\begin{equation}
    \dot{v} = F - \gamma v, \qquad \dot{r} 
    = v, \qquad \dot{\gamma} = \frac{1}{Q}\bigg(2K - 3NT_0\bigg)
\end{equation}


The integration scheme is now
\begin{align}
v_{i+1/2} 
    &= v_{i} + (F_i - \gamma_i v_i)\frac{\Delta t}{2} \\
\gamma_{i+1/2} 
    &= \gamma_i + \frac{1}{Q}(2K_i - 3NT_0)\frac{\Delta t}{2}. \\
r_{i+1} 
    &= r_{i} + v_{i+1/2}\Delta t \\
v_{i+1} 
    &= v_{i+1/2} + (F_{i+1} - \gamma_{i+1} v_{i+1/2})\frac{\Delta t}{2} \\
\gamma_{i+1} 
    &= \gamma_{i+1/2} + \frac{1}{Q}(2K_{i+1} - 3NT_0)\frac{\Delta t}{2}.	
\end{align}

The thermostat has now been included in the integration scheme
which makes it an implicit scheme---unlike the other thermostats
which do not rely on a particular scheme.

\section{Conclusion}
We first looked at two thermostats that are easy to understand
and implement.

The Berendsen thermostat is easy to use, but as we saw it does
not sample the canonical ensemble correctly. It should
therefore only be used for initialization.

The Andersen thermostat on the other hand samples the canonical ensemble.
It is also easy to use, but it is a stochastic method
and thus not time-reversible or deterministic.
The thermostat can give good results, but it also disturbs the dynamics
of the system, and messes up the trajectory.
It should therefore not be used for measuring time-dependent
quantities such as diffusion.

The aim of the project was to study the more complicated Nosé-Hoover
thermostat. It is unlike the two others both deterministic and
it samples the canonical ensemble.
It is harder to implement and understand, and it turns
the equations of motion into a (semi-) implicit scheme.


\end{document}
